\documentclass[]{beamer}
\usepackage{amsmath, amssymb}
\usepackage{graphicx}
\usepackage{multirow}
\usepackage{colortbl}
\usepackage{arydshln}
\usepackage{tikz} 
\usepackage{color}
\usepackage{textpos} 
\usepackage{booktabs}   			% For nicer tables
\usepackage[comma]{natbib} 	
\usepackage{bbm}					% Allows proper indicator functions: \mathbbm{1}
%\usepackage{multirow}
%\usepackage{dcolumn}
%\usepackage{babel}
\setbeamertemplate{navigation symbols}{}

%%%%%%%%% Packages and definitions for theorems
\usepackage{amsthm}
\theoremstyle{plain}
\newtheorem*{theorem*}{\protect\theoremname}
%\newtheorem{assumption}[theorem]{\protect\assumptionname}
\providecommand{\theoremname}{Theorem}

%%%%%%%%% For making pretty boxed propositions/theorems/assumptions etc
\usepackage[framemethod=TikZ]{mdframed}
% Increase the counter by one and make it available to ref
\newcounter{theo}[section]\setcounter{theo}{0}
%\renewcommand{\thetheo}{\arabic{section}.\arabic{theo}} % If we want section in thm number
\renewcommand{\thetheo}{\arabic{theo}} % If we don't want section in thm number
\newenvironment{boxedprop}[2][]{%
\refstepcounter{theo}%
\ifstrempty{#1}% % Checking if the title is empty
{\mdfsetup{% % This is what happens if the title is empty
		frametitle={%
			\tikz[baseline=(current bounding box.east),outer sep=0pt]
			\node[anchor=east,rectangle,fill=blue!20]
			{\strut Theorem};}}
			%{\strut Proposition~\thetheo};}} % Use this line if you want numbering
}%
{\mdfsetup{% % This is what happens if the title is not empty
		frametitle={%
			\tikz[baseline=(current bounding box.east),outer sep=0pt]
			\node[anchor=east,rectangle,fill=blue!20]
			{\strut Theorem:~#1};}}%
			%{\strut Proposition~\thetheo:~#1};}}% Use this line if you want numbering
}%
\mdfsetup{innertopmargin=-5pt,linecolor=blue!20,%
	linewidth=1pt,topline=true,%
	frametitleaboveskip=\dimexpr-\ht\strutbox\relax
}
\begin{mdframed}[]\relax%
\label{#2}}{\end{mdframed}}

%%%%%%%%%%%%%%%%%%%%%%%%%%%%%%%%%%%%%%%%%%%%%%%%%%%%%%%%%%%%%%%%%%%%%%%%%%%%%%%%%
% Commands for Light vs Dark Backgrounds
%%%%%%%%%%%%%%%%%%%%%%%%%%%%%%%%%%%%%%%%%%%%%%%%%%%%%%%%%%%%%%%%%%%%%%%%%%%%%%%%%
% Call the command right before the slide (outside of the frame envinronemt). Will
%    persist until the other command is called.

\newcommand{\UseLightBackground}{
	\usebackgroundtemplate{
		\tikz[overlay,remember picture] \node[opacity=1.6, at=(current page.center)] 
		{ \includegraphics[height=\paperheight,width=\paperwidth]{img/Background2.png} };
	}
}
\newcommand{\UseDarkBackground}{
	\usebackgroundtemplate{
		\tikz[overlay,remember picture] \node[opacity=1.6, at=(current page.center)] 
		{ \includegraphics[height=\paperheight,width=\paperwidth]{img/Background.png} };
	}
}

%%%%%%%%%%%%%%%%%%%%%%%%%%%%%%%%%%%%%%%%%%%%%%%%%%%%%%%%%%%%%%%%%%%%%%%%%%%%%%%%%
% Creates Section Signposts
%%%%%%%%%%%%%%%%%%%%%%%%%%%%%%%%%%%%%%%%%%%%%%%%%%%%%%%%%%%%%%%%%%%%%%%%%%%%%%%%%
%%%%% Can disable for a section using "\NextSectionWithoutTitlePage"

\newif\ifSectionTitlePage
\newcommand\NextSectionWithoutTitlePage{\SectionTitlePagefalse}
\SectionTitlePagetrue
\AtBeginSection[]
{
	\ifSectionTitlePage
%	\UseLightBackground  % Use the BLS background
%	\setbeamertemplate{footline}[reg page number]
	\begin{frame}
		\frametitle{Presentation Outline}
		\tableofcontents[currentsection]
	\end{frame}
	\fi
	\SectionTitlePagetrue % Turns section pages on for subsequent
}
\AtBeginSubsection[]
{
	\begin{frame}
		\frametitle{Table of Contents}
		\tableofcontents[currentsection,currentsubsection]
	\end{frame}
}

%%%%%%%%%%%%%%%%%%%%%%%%%%%%%% BLS template style changes
\definecolor{mygray}{gray}{0.30} % Defining my gray color for footnotes
\definecolor{navyblue}{rgb}{0.0, 0.0, 0.5}
\defbeamertemplate{footline}{left page number}
{%
  \hfill%
  \usebeamercolor[white]{page number in head/foot}%
  \usebeamerfont{page number in head/foot}%
  \insertframenumber\,/\,\inserttotalframenumber\kern53.5em\vskip10pt%
}
\setbeamertemplate{footline}[left page number]
\defbeamertemplate{footline}{reg page number}
{%
  \hfill%
  \usebeamercolor[navyblue]{page number in head/foot}%
  \usebeamerfont{page number in head/foot}%
  \insertframenumber\,/\,\inserttotalframenumber\kern53.5em\vskip10pt%
}

%%%%%%%%%%%%%%%%%%%%%%%%%%%%%%%%%%%%%%%%%%%%%%%%%%%%%%%%%%%%%%%%%%%%%%%%%%%%%%%%%
% Main Presentation Slides 
%%%%%%%%%%%%%%%%%%%%%%%%%%%%%%%%%%%%%%%%%%%%%%%%%%%%%%%%%%%%%%%%%%%%%%%%%%%%%%%%%

\begin{document}
\title{\textbf{\textcolor{white}{****** ADD TITLE HERE ******}}}
\author{\textbf{\textcolor{white}{** AUTHOR HERE **}}}
\institute{\normalsize{\textcolor{white}{U.S. Bureau of Labor Statistics \\ Office of Compensation and Working Conditions}}}
\date{\textcolor{white}{** PRESENTATION VENUE **\\ \vspace{.05cm} ** PRESENTATION DATE **}}

\UseDarkBackground
\begin{frame}[t]\textcolor{white}{
  \titlepage
}\end{frame}

\UseLightBackground
\setbeamertemplate{footline}[reg page number]
\begin{frame}{Disclaimer}
	\begin{center}
		The views expressed herein are those of the author(s) and do not necessarily reflect those of the Federal Government, Department of Labor, or the Bureau of Labor Statistics. 
		All results have been reviewed to ensure that no confidential information is disclosed.
	\end{center}
\end{frame}

\section{Introduction}

\begin{frame}
	Introduce, motivate, get the audience on board with what you're going to show them.
\end{frame}

\begin{frame}{Research Question}
	So what is the actual question/goal that you are talking about?
\end{frame}

\begin{frame}{Preview of Results}
	Give them a preview of what's to come in case they tune out for the rest of the presentation.
\end{frame}

\begin{frame}{Literature and Contribution}
	Maybe you need to talk about the literature? Give some context? Or maybe not, do whatever you want, I'm not your mother.
\end{frame}

\section{Model}

\begin{frame}{Model that Models All Models}
	Talk about the model with a bit of math, bit of description, bit of hand waving.
\end{frame}

\NextSectionWithoutTitlePage % Don't show the main section title page (just show the subsection one)
\section{Identification, Estimation, and Inference}

\subsection{Situation 1}

\begin{frame}{Situation 1}
	Mention some things that are specific to situation 1.
\end{frame}

\subsection{Situation 2}

\begin{frame}[label=situation2]{Situation 2}
	Surprise, things are different in situation 2! Who would have thought.
	
	Click \hyperlink{extra_details}{\beamerbutton{this button}} for more details in an appendix slide.
\end{frame}

\section{Monte Carlo Simulations}

\begin{frame}{Monte Carlo Simulations}
	Some pretty pictures and tables showing that, oh my god, the proposed methods actually work in a super idealized environment.
\end{frame}

\section{Application}

\begin{frame}{Application: Something in the Real World}
	This is what happens when we use the methods on a real world scenario.
\end{frame}

\begin{frame}{Conclusion}
	Here we conclude, summarize the project, etc...
\end{frame}

%}

\UseDarkBackground
\setbeamertemplate{footline}[left page number]

\begin{frame}\frametitle{\textcolor{white}{CONTACT INFORMATION}}
	\begin{center}
		\textcolor{white}{\huge{Thank You!}}
		\bigskip
		
		\textcolor{white}{\huge\textbf{Elan Segarra}}
		
		\textcolor{white}{\large U.S. Bureau of Labor Statistics \\ Office of Compensation and Working Conditions \\ Segarra.Elan@bls.gov}
	\end{center}

\end{frame}

%%%%%%%%%%%%%%%%%%%%%%%%%%%%%%%%%%%%%%%%%%%%%%%%%%%%%%%%%%%%%%%%%%%%%%%%%%%%%%%%%
% APPENDIX 
%%%%%%%%%%%%%%%%%%%%%%%%%%%%%%%%%%%%%%%%%%%%%%%%%%%%%%%%%%%%%%%%%%%%%%%%%%%%%%%%%

\appendix
\begin{frame}[allowframebreaks, noframenumbering]{Bibliography}
	\bibliographystyle{apa}
	\textcolor{white}{\bibliography{bib_file_name}}  % The bib file(s)
\end{frame}

\UseLightBackground
\setbeamertemplate{footline}[left page number]

\begin{frame}[label=extra_details]{Something Extra}
	Something a little extra here, just in case more details are required (or more time needs to be killed)
	
	Back to \hyperlink{situation2}{\beamerbutton{main slide}}.
\end{frame}

\end{document}