\documentclass[]{beamer}

%%%%%%%%%%%%%%%%%%%%%%%%%%%%%%%%%%%%%%%%%%%%%%%%%%%%%%%%%%%%%%%%%%%%%%%%%%%%%%%%%%%%%%%%%%%%%%%
% Misc Useful Settings and packages
%%%%%%%%%%%%%%%%%%%%%%%%%%%%%%%%%%%%%%%%%%%%%%%%%%%%%%%%%%%%%%%%%%%%%%%%%%%%%%%%%%%%%%%%%%%%%%%
%\usepackage{tikz} 							% For creating network graphs (among other things)
%\usepackage{booktabs}   					% For nicer tables
%\usepackage{multirow}						% For creating table cells that span multiple rows
%\usepackage{bbm}							% Allows proper indicator functions: \mathbbm{1}
%\setbeamertemplate{navigation symbols}{} 	% Supresses navigation symbols
\usepackage{textpos} 						% ?Boxes?

\setbeamertemplate{itemize items}[circle] 	% Changes bullets to circle (in itemize)
\setbeamertemplate{enumerate items}[circle] % Changes numbers to be put in circles (in enumerate)

%%%%%%%%%%%%%%%%%%%%%%%%%%%%%%%%%%%%%%%%%%%%%%%%%%%%%%%%%%%%%%%%%%%%%%%%%%%%%%%%%%%%%%%%%%%%%%%
% A: Packages for Theorems
%%%%%%%%%%%%%%%%%%%%%%%%%%%%%%%%%%%%%%%%%%%%%%%%%%%%%%%%%%%%%%%%%%%%%%%%%%%%%%%%%%%%%%%%%%%%%%%
\usepackage{amsthm}
\theoremstyle{plain}
\newtheorem*{theorem*}{\protect\theoremname}
%\newtheorem{assumption}[theorem]{\protect\assumptionname}
\providecommand{\theoremname}{Theorem}

%\usepackage{amssymb}                       % Includes \blacksquare
%\renewcommand{\qedsymbol}{$\blacksquare$}  % Changes the proof QED box to be a black square

%%%%%%%%%%%%%%%%%%%%%%%%%%%%%%%%%%%%%%%%%%%%%%%%%%%%%%%%%%%%%%%%%%%%%%%%%%%%%%%%%%%%%%%%%%%%%%%
% B: Pretty Boxes Around Propositions/Theorems/Assumptions
%%%%%%%%%%%%%%%%%%%%%%%%%%%%%%%%%%%%%%%%%%%%%%%%%%%%%%%%%%%%%%%%%%%%%%%%%%%%%%%%%%%%%%%%%%%%%%%
\usepackage[framemethod=TikZ]{mdframed}
% Increase the counter by one and make it available to ref
\newcounter{theo}[section]\setcounter{theo}{0}
%\renewcommand{\thetheo}{\arabic{section}.\arabic{theo}} % If we want section in thm number
\renewcommand{\thetheo}{\arabic{theo}} % If we don't want section in thm number
\newenvironment{boxedprop}[2][]{%
	\refstepcounter{theo}%
	\ifstrempty{#1}% % Checking if the title is empty
	{\mdfsetup{% % This is what happens if the title is empty
			frametitle={%
				\tikz[baseline=(current bounding box.east),outer sep=0pt]
				\node[anchor=east,rectangle,fill=blue!20]
				{\strut Theorem};}}
		%{\strut Proposition~\thetheo};}} % Use this line if you want numbering
	}%
	{\mdfsetup{% % This is what happens if the title is not empty
			frametitle={%
				\tikz[baseline=(current bounding box.east),outer sep=0pt]
				\node[anchor=east,rectangle,fill=blue!20]
				{\strut Theorem:~#1};}}%
		%{\strut Proposition~\thetheo:~#1};}}% Use this line if you want numbering
	}%
	\mdfsetup{innertopmargin=-5pt,linecolor=blue!20,%
		linewidth=1pt,topline=true,%
		frametitleaboveskip=\dimexpr-\ht\strutbox\relax
	}
	\begin{mdframed}[]\relax%
		\label{#2}
}{\end{mdframed}}

%%%%%%%%%%%%%%%%%%%%%%%%%%%%%%%%%%%%%%%%%%%%%%%%%%%%%%%%%%%%%%%%%%%%%%%%%%%%%%%%%
% C: Tikz Elements
%%%%%%%%%%%%%%%%%%%%%%%%%%%%%%%%%%%%%%%%%%%%%%%%%%%%%%%%%%%%%%%%%%%%%%%%%%%%%%%%%
\usepackage{tikz}
% This lets you define points relative to slides (see slide C1)
\tikzset{
	use page relative coordinates/.style={
		shift={(current page.south west)},
		x={(current page.south east)},
		y={(current page.north west)}
	},
}

% This lets you add "visible on" property to tikz elements (see slide C2)
\usetikzlibrary{overlay-beamer-styles}


\begin{document}
	
\begin{frame}{B: Pretty Boxed Theorems}
	Below is an example of a boxed theorem environment.
	
	\begin{theorem}[additional text]
		Let $A$ and $B$ be $ n\times n $ matrices of real numbers. If $ A $ and $ B $ are both invertible, then the matrix $ AB $ is invertible.
	\end{theorem}
\end{frame}

\begin{frame}{C1: Tikz Elements (Add Overlayed Arrows!)}
	Below is an example of using tikz to create arrows for pointing at things on the frame.
	
	\vspace{0.65\paperheight} % Creating space between above and below text.
	
	Keep in mind this method ignores whatever is on the slide, so you may need to add invisible space so that text accommodates.
	
	\begin{tikzpicture}[remember picture,overlay,use page relative coordinates]
		% Coords are relative to SW corner so (X,Y) is X% from left side and Y% up from bottom
		\node (a) at (0.1,0.2) {Who?};
		\node (b) at (0.5,0.8) {What?};
		\node (c) at (0.8,0.5) {Where?};
		\node (x) at (0.5,0.5) {Something!};
		\draw[-stealth] (a) -- (x);
		\draw[-stealth] (b) -- (x);
		\draw[-stealth] (c) -- (x);
	\end{tikzpicture}
\end{frame}

\begin{frame}{C2: Tikz Elements (Make Elements Appear Over Time!)}
	Below is an example of making the tikz elements appear over time as you click through.
	
	\vspace{0.65\paperheight} % Creating space between above and below text.
	
	Keep in mind this method ignores whatever is on the slide, so you may need to add invisible space so that text accommodates.
	
	\begin{tikzpicture}[remember picture,overlay,use page relative coordinates]
		% Coords are relative to SW corner so (X,Y) is X% from left side and Y% up from bottom
		\node (a) at (0.1,0.2) {Who?};
		\node[visible on=<2->] (b) at (0.5,0.8) {What?};
		\node[visible on=<3->] (c) at (0.8,0.5) {Where?};
		\node (x) at (0.5,0.5) {Something!};
		\draw[-stealth] (a) -- (x);
		\draw[-stealth,visible on=<2->] (b) -- (x);
		\draw[-stealth,visible on=<3->] (c) -- (x);
	\end{tikzpicture}
\end{frame}
	
\end{document}